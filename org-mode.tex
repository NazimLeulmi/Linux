% Created 2018-08-19 Sun 16:13
% Intended LaTeX compiler: pdflatex
\documentclass[11pt]{article}
\usepackage[utf8]{inputenc}
\usepackage[T1]{fontenc}
\usepackage{graphicx}
\usepackage{grffile}
\usepackage{longtable}
\usepackage{wrapfig}
\usepackage{rotating}
\usepackage[normalem]{ulem}
\usepackage{amsmath}
\usepackage{textcomp}
\usepackage{amssymb}
\usepackage{capt-of}
\usepackage{hyperref}
\author{ryan}
\date{\today}
\title{A Quick Introduction to Org-mode}
\hypersetup{
 pdfauthor={ryan},
 pdftitle={A Quick Introduction to Org-mode},
 pdfkeywords={},
 pdfsubject={},
 pdfcreator={Emacs 25.2.2 (Org mode 9.1.13)}, 
 pdflang={English}}
\begin{document}

\maketitle
\tableofcontents

\section*{what is org-mode ?}
\label{sec:orgdec7fb4}
\begin{itemize}
\item It is a mode within emacs that started its life as an outliner .
\item org mode is usually used as a markup language to write notes ,to do lists , and much more.
\item It is an example of an exocortex , which is something to rely on other than your brain to remember things.
\end{itemize}
\section*{org-mode features}
\label{sec:org53d0999}
\begin{itemize}
\item Easy outlining with \texttt{tab} folding to maximize focus on the task.
\item Task management \& to-do lists with different states
\begin{itemize}
\item \texttt{t} to cycle through the different states of the list item.
\end{itemize}
\end{itemize}
\subsubsection*{To do lists must be headers}
\label{sec:org4e9d305}
\begin{itemize}
\item {\bfseries\sffamily DONE} task uno
\label{sec:orgf62d7a0}
\item {\bfseries\sffamily TODO} 2nd task
\label{sec:org7b27fd7}
\item {\bfseries\sffamily TODO} 3rd  task
\label{sec:org6842279}
\end{itemize}
\subsubsection*{Progress indicators [2/3]}
\label{sec:orge418c8f}
\begin{itemize}
\item {\bfseries\sffamily TODO} barber
\label{sec:orga622b64}
\item {\bfseries\sffamily DONE} walk dog
\label{sec:org72b6379}
\item {\bfseries\sffamily DONE} launch
\label{sec:org0999887}
\end{itemize}
\subsubsection*{Percentage indicators [75\%]}
\label{sec:org4d21d18}
\begin{itemize}
\item {\bfseries\sffamily DONE} wash car
\label{sec:org8431209}
\item {\bfseries\sffamily DONE} cook dinner
\label{sec:org9f8fb50}
\item {\bfseries\sffamily TODO} run
\label{sec:org137f0b3}
\item {\bfseries\sffamily DONE} shower
\label{sec:org64102ee}
\end{itemize}

\section*{Markup}
\label{sec:org851b625}
\subsubsection*{Text Transformations}
\label{sec:org0680bed}
\begin{itemize}
\item \emph{italic} text
\item \textbf{bold} text
\item \texttt{verbatim}
\item \sout{stroke through} text
\end{itemize}
\subsubsection*{Meta Data}
\label{sec:orga16078f}
\begin{itemize}
\item Title
\item Options [disable/enable] features
\end{itemize}
\subsubsection*{Links \texttt{[[address][description]]}}
\label{sec:orga48f3cb}
\begin{itemize}
\item C-c + C-l to create a link automatically
\item HTTP links
\href{https://github.com//ryanLeulmi}{my github}
\item Local Links to files / todos / files / etc..
\href{file:///home/ryan/.spacemacs}{dotspacemacs}
\item C-c + C-o to open a local link in a new buffer
\end{itemize}
\subsubsection*{Source Code}
\label{sec:org8433b50}
\begin{itemize}
\item \texttt{<s + tab} to create a code block
\item \texttt{C-c + '} to open a code block in a new buffer with it's mode
\end{itemize}
\begin{verbatim}
export PATH="`yarn global bin`:$PATH"
\end{verbatim}
\subsubsection*{Raw HTML export}
\label{sec:org69594fa}
\begin{verbatim}
#+BEGIN_EXPORT html
<h3> RAW HTML </h3>
#+END_EXPORT
\end{verbatim}
\subsubsection*{Smarter Tables}
\label{sec:org9ccc183}
\begin{center}
\begin{tabular}{lll}
Student \# & Email & Password\\
\hline
ryan & ryanleulmi@gmail.com & 1234\\
student & another one & asd21\\
\end{tabular}
\end{center}
\section*{Export to other formats from the same source}
\label{sec:org500ad24}
\begin{itemize}
\item C-c + C-e
\item twitter bootstrap \texttt{ox-twbs}
\item html , pdf , markdown \texttt{ox-gfm}
\item \texttt{org-export-backends} to manage your export backends.
\item \texttt{org-file-apps} to change which apps your export backends use.
\end{itemize}
\end{document}
